\newpage
\section{Ejercicio 2}

En este caso, se analiza como seria el programa, si se agregasen fotos al mapa.

El problema que se presenta es que el Interval Tree esta pensado para, una vez creado, no ser modificado. Al intentar ingresar un intervalo nuevo, este podria agrandar el rango, y asi no quedaria la idea original plasmada, ya que ya no seria el pivot el elemento elegido.



Para resolverlo, se propone, en la funcion "Insertar Foto", crear nuevamente ambos arboles de intervalos, conteniendo los nuevos intervalos, para que se balancee nuevamente y asi no modificar su complejidad.

Otra alternativa, es, por ejemplo en el caso en que la nueva foto no modifique el rango total, ir viendo nodo por nodo para ver en cual se puede insertar, y ahi ponerlo en las listas, ordenandolas nuevamente.



En el primer caso, la complejidad seria la de crear otra vez los dos arboles, o sea O($n\log_2(n)$).

En el segundo, la complejidad seria similar a la busqueda, recorriendo el arbol, y una vez encontrado el nodo indicado, agregar a las listas y ordenar. En este caso es O($\log_2(n) + 2n\log_2(n)$). Se remarca que en la parte de ordenar, la lista va a estar practicamente ordenada, lo que achica, en practica, la complejidad del algoritmo.


Una vez realizado esto, la interfaz superior del programa no se modifica, a excepci'on de agregar la funcion insertar. El resto de las complejidades se mantienen como son.

Hay que remarcar que esta es una variante al Interval Tree, ya que este no esta pensado, en su version standard, para realizar inserciones.

