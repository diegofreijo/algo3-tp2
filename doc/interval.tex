\newpage
\section{Interval Tree}

%%%%%%%%%%%%%%%%%%%%%%%%%%%%%%%%%%%%%%%%%%%%%%%%%%%%%%%%%%%%%%%%%%%%%%
% Pseudocodigos
%%%%%%%%%%%%%%%%%%%%%%%%%%%%%%%%%%%%%%%%%%%%%%%%%%%%%%%%%%%%%%%%%%%%%%
\clearpage
\subsection{Pseudoc'odigos}

La creaci'on del Interval Tree es de la siguiente forma:

Se calcula el rango de las fotos, para saber el menor inicio y mayor final. Una vez calculado esto, se procede a tomar un punto cercano al medio. Con este punto como pivot, se generan tres listas. Una con aquellos intervalos que contienen al pivot, llamada centro; otra con los intervalos que estan completamente a la izquierda, llamada izquierda; y por ultimo otra con los que estan a la derecha del punto, llamada derecha.

Para ver a que lista pertenece, se utilizan las funciones Contiene, Anterior y Posterior. La primera verifica si el pivot esta entre el inicio y el fin del intervalo. La segunda, si el final del intervalo es menor al punto, entonces, esta a la izquierda del mismo. Y la tercera, si el inicio es mayor al punto, el intervalo, entonces, es posterior.
Una vez generadas estas listas, se crea el nodo con: el pivot utilizado para comparar y la lista Centro. 

El nodo genera internamente sus miembros: el pivot se mantiene, la lista Inicio es generada ordenando a la lista Centro por sus inicios, y la lista Fin ordenando a la misma lista, pero por sus finales.
Al estar creado el nodo, se inserta en una lista.

Para ordenar se utiliza el algoritmo QuickSort, al ser un algoritmo eficiente en caso promedio. Se eligio sobre otras opciones como el SelectionSort, BubbleSort o InsertionSort, ya que estos son peores en la practica al elegido.

El procedimiento se vuelve a repetir con la listas Izquierda y Derecha, agregando los nodos a la misma lista.

Al terminar este proceso recursivo, se agregan los nodos al arbol RB.

\vspace{2em}


%%%%%%%%%%%%%%%%%%%%%%%%%%%%%%%%%%%%%%%%%%%%%%%%%%%%%%%%%%%%%%%%%%%%%%
% Analisis de complejidad
%%%%%%%%%%%%%%%%%%%%%%%%%%%%%%%%%%%%%%%%%%%%%%%%%%%%%%%%%%%%%%%%%%%%%%
\subsection{Analisis de complejidad - Modelo Uniforme}
{\bf\small Temporal}
\subsection{Creaci'on}
El peor caso para la creaci'on del Interval es que todos los intervalos sean disjuntos, lo que provocar'ia que en la lista Centro de cada nodo, haya solo un intervalo. Vamos a tomar este caso para analizar la complejidad.

La complejidad de las funciones Anterior, Posterior y Contiene es O(1), ya que solo realiza una comparaci'on y devuelve true o false.

La complejidad de BuscoRango es O(n), ya que recorre todos los intervalos para encontrar el menor inicio y el mayor fin. Dentro del while, las operaciones son O(1).

Dentro del algoritmo del constructor, se llama una vez a la funcion BuscoRango y se repiten n veces las funciones Anterior, Posterior y Contiene. Luego, se repite recursivamente $\log_2(n)$ veces. Por lo tanto la complejidad es del orden $n \log_2(n)$.

El tama�o de la entrada se calcula de la siguiente manera. La entrada contiene la cantidad de fotos, n, y sus coordenadas. Los valores de las coordenadas estan acotadas por el tama�o de la pantalla, lo que hace que sea constante, por lo cual, el tama�o de la entrada queda $\log_2(n)+n$, que es del orden de n.

Por lo cual, la complejidad de la creaci'on queda O($\log_2(n)+n$).

{\bf\small Espacial}

La complejidad espacial de la creaci'on del Interval Tree es la siguiente: en cada nodo se guardan 2 enteros y 2 listas iguales de enteros con la cantidad de fotos que tocan a ese punto. Esto se repite, en peor caso, n veces. Por lo tanto, el orden de complejidad del algoritmo es O($n$).

 \clearpage
