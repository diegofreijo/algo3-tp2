\documentclass[spanish, a4paper, 11pt]{article}

\usepackage[a4paper,margin=3.5cm,top=3.0cm,bottom=3.0cm]{geometry}	% Define los margenes
\usepackage[spanish,activeacute]{babel}								% Idioma castellano
\usepackage{caratula}														% Caratula de Algo2
%\usepackage[a4paper=true,pagebackref=true]{hyperref}				% Agrega la TOC al PDF e hipervinculos
%\usepackage[pdftex]{graphicx} 											% Permite insertar graficos
\usepackage{fancyhdr}														% Permite manejo de cabeceras de pagina
\usepackage{eufrak}															% Usado en el enunciado del trabajo
\usepackage{latexsym}
\usepackage{algorithmic}													% Para escribir los algos
\usepackage{dsfont}															% Para el simbolo de naturales


% Estilo de pagina para tener las cabeceras
\pagestyle{fancy}
\lhead{AyED III - TP2}
\rhead{Matias Blanco - Diego Freijo - Hernan Guebel - Marcos Rogani}

% Numeracion de paginas
\pagenumbering{arabic}
\parskip=1.5ex

% Seteo de estilo de los algoritmos
\algsetup{indent=2em}

\newcommand{\imagen}[3]
{
	\begin{figure}[h]
	  \centering
	%    \includegraphics[width=#2cm]{#1}
	  \caption{#3}
	\end{figure}
}

\newcommand{\nat}{\mathds{N}}
\newcommand{\algoritmo}[3]{\noindent {\bf\underline{#1}:} #2 $\longrightarrow$ #3}
\newcommand{\superindice}[1]{$^\textrm{{\tiny #1}}$}
\newcommand{\subsubsubsection}[1]{

{\bf\small #1}

}
\newcommand{\negrita}[1]{{\bf #1}}

%%%%%%%%%%%%%%%%%%%%%%%%%%%%%%%%%%%%%%%%%%%%%%%%%%%%%%%%%%%%%%
%%%%%%%%%%%%%%%%%%%%%%%%%%%%%%%%%%%%%%%%%%%%%%%%%%%%%%%%%%%%%%%%%%%%%%%%%%%%%%%%%%%%
%%%%%   Inicio del documento
%%%%%%%%%%%%%%%%%%%%%%%%%%%%%%%%%%%%%%%%%%%%%%%%%%%%%%%%%%%%%%%%%%%%%%%%%%%%%%%%%%%%
%%%%%%%%%%%%%%%%%%%%%%%%%%%%%%%%%%%%%%%%%%%%%%%%%%%%%%%%%%%%%%

\begin{document}

%%%%%%%%%%%%%%%%%%%%%%%%%%%%%%%%%%%%%%%%%%%%%%%%%%%%%%%%%%%%%%%%%%%%%%
% Caratula
%%%%%%%%%%%%%%%%%%%%%%%%%%%%%%%%%%%%%%%%%%%%%%%%%%%%%%%%%%%%%%%%%%%%%%
\materia{Algoritmos y Estructuras de Datos III}
\submateria{Primer Cuatrimestre de 2007}
\titulo{Trabajo Pr'actico 2}
\subtitulo{}
\integrante{Blanco, Matias}{508/05}{matiasblanco18@gmail.com}
\integrante{Freijo, Diego}{4/05}{giga.freijo@gmail.com}
\integrante{Guebel, Hernan}{??/??}{hguebel@gmail.com}
\integrante{Rogani, Marcos}{??/??}{marcos.rogani@gmail.com}
\maketitle

\subsection*{Palabras Clave}
Red-Black Trees, Interval Trees.

%%%%%%%%%%%%%%%%%%%%%%%%%%%%%%%%%%%%%%%%%%%%%%%%%%%%%%%%%%%%%%%%%%%%%%
% Indice
%%%%%%%%%%%%%%%%%%%%%%%%%%%%%%%%%%%%%%%%%%%%%%%%%%%%%%%%%%%%%%%%%%%%%%
\clearpage
\tableofcontents
\clearpage

%%%%%%%%%%%%%%%%%%%%%%%%%%%%%%%%%%%%%%%%%%%%%%%%%%%%%%%%%%%%%%%%%%%%%%
% Ejercicios
%%%%%%%%%%%%%%%%%%%%%%%%%%%%%%%%%%%%%%%%%%%%%%%%%%%%%%%%%%%%%%%%%%%%%%
%\input{ej1.tex}



%%%%%%%%%%%%%%%%%%%%%%%%%%%%%%%%%%%%%%%%%%%%%%%%%%%%%%%%%%%%%%%%%%%%%%
% Enunciado
%%%%%%%%%%%%%%%%%%%%%%%%%%%%%%%%%%%%%%%%%%%%%%%%%%%%%%%%%%%%%%%%%%%%%%
%\newpage
%\section*{Enunciado}
%----------------------------------------------------------------------------------%
%??????????????????
%----------------------------------------------------------------------------------%
%\newpage


%%%%%%%%%%%%%%%%%%%%%%%%%%%%%%%%%%%%%%%%%%%%%%%%%%%%%%%%%%%%%%%%%%%%%%
% Firmas
%%%%%%%%%%%%%%%%%%%%%%%%%%%%%%%%%%%%%%%%%%%%%%%%%%%%%%%%%%%%%%%%%%%%%%
% ????????

%%%%%%%%%%%%%%%%%%%%%%%%%%%%%%%%%%%%%%%%%%%%%%%%%%%%%%%%%%%%%%%%%%%%%%
% Referencias
%%%%%%%%%%%%%%%%%%%%%%%%%%%%%%%%%%%%%%%%%%%%%%%%%%%%%%%%%%%%%%%%%%%%%%
\section{Referencias}
\begin{thebibliography}{99}

	\bibitem{strassen} Algoritmo de Strassen para la multiplicaci'on de matrices cuadradas:\\  {\it http://en.wikipedia.org/wiki/Strassen\_algorithm}

	\bibitem{cw} Algoritmo de Coppersmith-�Winograd para la multiplicaci'on de matrices cuadradas:\\  {\it http://en.wikipedia.org/wiki/Coppersmith\%E2\%80\%93Winograd\_algorithm}

	\bibitem{binpow} Golub \& Van Loan, {\it Matrix Computations}; secci'on 11.2.5, p'agina 569.

\end{thebibliography}



\end{document}
%%%
% EOF
%%%%%%%%%%%%%%%%%%%%%
