\section{An'alisis de complejidad}
\subsection{Creaci'on}
\subsubsection{Temporal}
El peor caso para la creaci'on del Interval es que todos los intervalos sean disjuntos, lo que provocar'ia que en la lista Centro de cada nodo, haya solo un intervalo. Vamos a tomar este caso para analizar la complejidad.

La complejidad de las funciones Anterior, Posterior y Contiene es O(1), ya que solo realiza una comparaci'on y devuelve true o false.

La complejidad de BuscoRango es O(n), ya que recorre todos los intervalos para encontrar el menor inicio y el mayor fin. Dentro del while, las operaciones son O(1).

Dentro del algoritmo del constructor, se llama una vez a la funcion BuscoRango y se repiten n veces las funciones Anterior, Posterior y Contiene. Luego, se repite recursivamente $\log_2(n)$ veces. Por lo tanto la complejidad es del orden $n \log_2(n)$.

El tama�o de la entrada se calcula de la siguiente manera. La entrada contiene la cantidad de fotos, n, y sus coordenadas. Los valores de las coordenadas estan acotadas por el tama�o de la pantalla, lo que hace que sea constante, por lo cual, el tama�o de la entrada queda $\log_2(n)+n$, que es del orden de n.

Por lo cual, la complejidad de la creaci'on queda O($\log_2(n)+n$).

\subsubsection{Espacial}

La complejidad espacial de la creaci'on del Interval Tree es la siguiente: en cada nodo se guardan 2 enteros y 2 listas iguales de enteros con la cantidad de fotos que tocan a ese punto. Esto se repite, en peor caso, n veces. Por lo tanto, el orden de complejidad del algoritmo es O($n$).



\subsection{BuscarInterseccion}
\subsubsection{Temporal}
O($log(n)+k$)

Siendo $n$ la cantidad de nodos totales del 'arbol de intervalos y k la cantidad de intervalos obtenidos en la b'usqueda. 

En realidad en cada nodo, como se explic'o anteriormente, si la cantidad de intervalos obtenidos es menor a la cantidad de intervalos de la lista se realizan $k+1$ comparaciones pero la constante 1 es despreciable. Si se devuelven todos los intervalos se realizan k comparaciones.

$log(n)$ ya que es un 'arbol balanceado y lo recorro desde la raiz hasta alguna hoja.

\subsubsection{Espacial}
O($n$)

Ya que cada segmento es guardado en dos listas y el 'arbol es balanceado.

\subsection{Fusionar}
\subsubsection{Temporal}
Como se explic'o anteriormente en la introducci'on del presente, en el peor de los casos 'este algoritmo deber'a recorrer 3 veces todas las im'agenes (caso en el que el punto est'e contendio en toda im'agen de la pantalla). Por lo que la complejidad ser'a O($3n$) = O($n$).

\subsubsection{Espacial}
El algoritmo s'olo requiere (sin contar el espacio para la lista de resultados) los bits de seleccionado en las im'agenes y las referencias a los padres en cada intervalo. Por lo que se requieren cantidad de n'umeros del 'orden de la cantidad de im'agenes, siendo la complejidad O($n$).

