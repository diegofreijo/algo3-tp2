\section{Pseudoc'odigos}
\algoritmo{Insertar($k$, $valor$)}{Inserta un nodo en el arbol (como si fuera un 'arbol binario de b'usqueda) y luego reestablece el invariante del RedBlackTree}{O($log(n)$)}
\begin{algorithmic}[1]
\STATE $elem.construir(k, valor)$
\STATE $y \leftarrow null$
\STATE $x \leftarrow this.root$
\WHILE{x != null}
	\STATE $y \leftarrow x$
	\IF{k $<$ x.key}
		\STATE $x \leftarrow x.izq$
	\ELSE
		\STATE $x \leftarrow x.der$
	\ENDIF
\ENDWHILE
\STATE $elem.padre \leftarrow y$
\IF{y $<$ null}
	\STATE $this.root \leftarrow elem$
\ELSIF{k $<$ y.key}
	\STATE $y.izq \leftarrow elem$
\ELSE
	\STATE $y.der \leftarrow elem$
\ENDIF
\STATE $incremento el tama�o del arbol en 1$
\STATE $reestablecer invariante$
\end{algorithmic}

\vspace{2em}

La creaci'on del Interval Tree es de la siguiente forma:

Se calcula el rango de las fotos, para saber el menor inicio y mayor final. Una vez calculado esto, se procede a tomar un punto cercano al medio. Con este punto como pivot, se generan tres listas. Una con aquellos intervalos que contienen al pivot, llamada centro; otra con los intervalos que estan completamente a la izquierda, llamada izquierda; y por ultimo otra con los que estan a la derecha del punto, llamada derecha.

Para ver a que lista pertenece, se utilizan las funciones Contiene, Anterior y Posterior. La primera verifica si el pivot esta entre el inicio y el fin del intervalo. La segunda, si el final del intervalo es menor al punto, entonces, esta a la izquierda del mismo. Y la tercera, si el inicio es mayor al punto, el intervalo, entonces, es posterior.
Una vez generadas estas listas, se crea el nodo con: el pivot utilizado para comparar y la lista Centro. 

El nodo genera internamente sus miembros: el pivot se mantiene, la lista Inicio es generada ordenando a la lista Centro por sus inicios, y la lista Fin ordenando a la misma lista, pero por sus finales.
Al estar creado el nodo, se inserta en una lista.

Para ordenar se utiliza el algoritmo QuickSort, al ser un algoritmo eficiente en caso promedio. Se eligio sobre otras opciones como el SelectionSort, BubbleSort o InsertionSort, ya que estos son peores en la practica al elegido.

El procedimiento se vuelve a repetir con la listas Izquierda y Derecha, agregando los nodos a la misma lista.

Al terminar este proceso recursivo, se agregan los nodos al arbol RB.

\vspace{2em}

\algoritmo{BuscarInterseccion($valorBusq$)}{Devuelve la lista de intervalos que contienen al valor buscado}{O($log(n)+k$)}
\begin{algorithmic}[1]
\WHILE{!salir}
	\IF{nodo.key $<$ $valorBusq$}
		\STATE agrego a la solucion los intervalos que terminan despues de $valorBusq$
		\IF{derecha no es $null$}
				\STATE voy para la derecha
		\ELSE
				\STATE salir
		\ENDIF
	\ELSIF{nodo.key $>$ $valorBusq$}
		\STATE agrego a la solucion los intervalos que empiezan antes de $valorBusq$
		\IF{izquierda no es $null$}
				\STATE voy para la izquierda
		\ELSE
				\STATE salir
		\ENDIF
	\ELSE
		\STATE agrego todos los intervalos a la solucion
		\IF{derecha no es $null$}
				\STATE voy para la derecha
		\ELSE
				\STATE salir
		\ENDIF
	\ENDIF
\ENDWHILE
\end{algorithmic}

\vspace{2em}

\algoritmo{dameTermDespues($nodo$, $valor$)}{Devuelve una lista con los intervalos del nodo que terminan despues de valor, utilizando la lista ordenada por el valor fin de mayor a menor}{O($+k$)}
\begin{algorithmic}[1]
\STATE $intervalos \leftarrow  nodo.fin$
\STATE $i \leftarrow 0$
\WHILE{i < cantidad(intervalos) $\land$ intervalos($i$).fin $>=$ valor }
	\STATE agregar intervalo a la solucion
 	\STATE $i++$
\ENDWHILE
\end{algorithmic}

\vspace{2em}

\algoritmo{dameEmpAntes($nodo$, $valor$)}{Devuelve una lista con los intervalos del nodo que empiezan antes de valor, utilizando la lista ordenada por el valor inicio de mayor a menor}{O($+k$)}
\begin{algorithmic}[1]
\STATE $intervalos \leftarrow  nodo.inicio$
\STATE $i \leftarrow cantidad(intervalos)-1$
\WHILE{i > 0 $\land$ intervalos($i$).inicio $<=$ valor }
	\STATE agregar intervalo a la solucion
 	\STATE $i--$
\ENDWHILE
\end{algorithmic}

\vspace{2em}

\algoritmo{Fusionar($p, arbol_x, arbol_y, imagenes$)}{En realidad llamado BuscarInterseccion, devuelve las im'agenes en $imagenes$ que contienen al punto $p$ utilizando los 'arboles}{O($n$)}
\begin{algorithmic}[1]
\STATE $intersecciones_x$ = $arbol_x$.BuscarInterseccion($p$)
\STATE $intersecciones_y$ = $arbol_y$.BuscarInterseccion($p$)
\FORALL{$int$ {\bf in} $intersecciones_x$}
	\STATE marcar al padre de $int$ en $x$
\ENDFOR
\FORALL{$int$ {\bf in} $intersecciones_y$}
	\STATE marcar al padre de $int$ en $y$
\ENDFOR
\FORALL{$img$ {\bf in} $imagenes$}
	\IF{$img$ est'a marcado en $x$ e $y$}
		\STATE $resultados$.agregar($img$)
	\ENDIF
	\STATE limpiar marcas de $img$
\ENDFOR
\RETURN $resultados$
\end{algorithmic}

