\newpage
\section{Busqueda en el Arbol de Intervalos}

%%%%%%%%%%%%%%%%%%%%%%%%%%%%%%%%%%%%%%%%%%%%%%%%%%%%%%%%%%%%%%%%%%%%%%
% Explicacion
%%%%%%%%%%%%%%%%%%%%%%%%%%%%%%%%%%%%%%%%%%%%%%%%%%%%%%%%%%%%%%%%%%%%%%
\subsection{Explicaci'on}
En cada nodo del 'arbol tengo guardadas 2 listas, con los mismos intervalos, pero ordenadas de diferente manera.
Un lista la tengo ordena de mayor a menor por el valor de inicio de cada intervalo.
La otra lista la tengo ordena, tambi'en de mayor a menor, pero por el valor de fin de cada intervalo.
El objetivo es dado un punto x encontrar todos los intervalos que contienen a ese punto. Para esto la idea del algoritmo consiste en ir comparando el valor x buscado con la clave de la ra�z del 'arbol: 
	Si el valor de la clave es menor al de x agrego a la solucion los intervalos  del nodo que terminan despu'es de x utilizando la lista ordenada de mayor a menor por el valor fin. Utilizo esta lista porque ya se que todos los intervalos del nodo arrancan antes de x y al estar ordenada tendr'e que hacer k+1 comparaciones (o hasta la cantidad de intervalos) para obtener los k intervalos que contienen a ese punto. Luego de agregar los intervalos, si la rama derecha no es nula continuo con el algoritmo, tomando como el subarbol izquierdo. Si es nula salgo.
	Si el valor de la clave es mayor al de x agrego a la solucion los intervalos del nodo que empiezan antes de x utilizando la lista ordenada de mayor a menor por el valor inicio. Utilizo esta lista porque ya se que todos los intervalos del nodo terminan despu�s de x, y al igual que en el otro caso, al estar ordenada tendr'e que hacer k+1 comparaciones (o hasta la cantidad de intervalos) para obtener los k intervalos que contienen a ese punto. Luego de agregar los intervalos, si la rama izquierda no es nula continuo con el algoritmo, tomando el sub�rbol derecho. Si es nula salgo.
	Si el valor buscado es igual al de la clave, agrego a la solucion todos los intervalos del valor del nodo y si la rama izquierda no es nula contin�o con el algoritmo, tomando el subarbol derecho. Si es nula salgo.


%%%%%%%%%%%%%%%%%%%%%%%%%%%%%%%%%%%%%%%%%%%%%%%%%%%%%%%%%%%%%%%%%%%%%%
% Pseudocodigos
%%%%%%%%%%%%%%%%%%%%%%%%%%%%%%%%%%%%%%%%%%%%%%%%%%%%%%%%%%%%%%%%%%%%%%
\clearpage
\subsection{Pseudoc'odigo}

\algoritmo{<BuscarInterseccion($valorBusq$)}{Devuelve la lista de intervalos que contienen al valor buscado}{O($log(n)+k$)}
\begin{algorithmic}[1]
\WHILE{!salir}
	\IF{ nodo.key $<$ $valorBusq$}
		\STATE $agrego a la solucion los intervalos que terminan despues de 'valorBusq'$ 
		\IF{derecha no es $null$}
				\STATE $voy para la derecha$
		\ELSE
				\STATE salir$
		\ENDIF
	\ELSEIF{ nodo.key $>$ $valorBusq$}
		\STATE $agrego a la solucion los intervalos que empiezan antes de 'valorBusq'$ 
		\IF{izquierda no es $null$}
				\STATE $voy para la izquierda$
		\ELSE
				\STATE salir$
		\ENDIF
	\ELSE
		\STATE $agrego todos los intervalos a la solucion$ 
		\IF{derecha no es $null$}
				\STATE $voy para la derecha$
		\ELSE
				\STATE salir$
		\ENDIF
	\ENDIF
\ENDWHILE
\end{algorithmic}


%%%%%%%%%%%%%%%%%%%%%%%%%%%%%%%%%%%%%%%%%%%%%%%%%%%%%%%%%%%%%%%%%%%%%%
% Analisis
%%%%%%%%%%%%%%%%%%%%%%%%%%%%%%%%%%%%%%%%%%%%%%%%%%%%%%%%%%%%%%%%%%%%%%
\subsection{Analisis}
\subsubsection{Complejidad}
O($log(n)+k$) 
Siendo n la cantidad de nodos totales del 'arbol de intervalos y k la cantidad de intervalos obtenidos en la b'usqueda. 
En realidad en cada nodo, como se explico anteriormente, se realizan $k+1$ comparaciones pero la constante 1 es despreciable. 
$log(n)$ ya que es un 'arbol balanceado y lo recorro desde la raiz hasta alguna hoja.


\subsubsection{Complejidad Espacial}
O($n) 
Ya que cada segmento es guardado en dos listas y el 'arbol es balanceado.

\clearpage
